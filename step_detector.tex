\documentclass[12pt,a4paper]{article}
\usepackage[UTF8]{ctex}
\usepackage{amsmath,amssymb}
\usepackage{graphicx}
\usepackage{booktabs}
\usepackage{geometry}
\usepackage{enumitem}
\usepackage{listings}
\usepackage{xcolor}
\usepackage{algorithm}
\usepackage{algpseudocode}

\geometry{margin=2.5cm}

\title{\textbf{基于IMU的计步算法分析与比较}}
\author{计步检测系统}
\date{\today}

\begin{document}

\maketitle

\section{概述}

本文档详细介绍三种基于惯性测量单元(IMU)加速度数据的计步算法:
\begin{enumerate}
    \item \textbf{峰值检测法} (Peak Detection)
    \item \textbf{过零检测法} (Zero-Crossing Detection)
    \item \textbf{自相关函数法} (Autocorrelation)
\end{enumerate}

三种方法共享相同的预处理流程,仅在最终计步算法上有所不同。

%========================================
\section{通用预处理流程}
%========================================

所有三种方法均采用以下预处理步骤:

\subsection{步骤1:数据加载}
从IMU传感器读取三轴加速度数据 $(a_x, a_y, a_z)$ 及对应时间戳。

\subsection{步骤2:计算合成加速度}
将三轴加速度合成为标量模值,消除设备方向的影响:
\begin{equation}
    a_{mag} = \sqrt{a_x^2 + a_y^2 + a_z^2}
\end{equation}

\subsection{步骤3:带通滤波}
应用4阶巴特沃斯带通滤波器,保留步态相关频率成分:
\begin{itemize}
    \item 低截止频率:$f_L = 0.5$ Hz(排除缓慢漂移)
    \item 高截止频率:$f_H = 5.0$ Hz(排除高频噪声)
    \item 正常步频范围:$0.5 \sim 3.0$ Hz(约每秒0.5到3步)
\end{itemize}

滤波器传递函数:
\begin{equation}
    H(s) = \frac{\omega_H^n \cdot s^n}{(s^2 + \frac{\omega_L}{Q}s + \omega_L^2)^{n/2} \cdot (s^2 + \frac{\omega_H}{Q}s + \omega_H^2)^{n/2}}
\end{equation}

使用 \texttt{scipy.signal.filtfilt} 进行零相位滤波,避免相位延迟。

%========================================
\section{方法一:峰值检测法 (Peak Detection)}
%========================================

\subsection{算法原理}
每一步行走会在加速度信号中产生一个明显的峰值。通过检测信号中的局部最大值来计数步数。

\subsection{算法步骤}

\begin{enumerate}
    \item \textbf{计算自适应阈值}:
    \begin{equation}
        \text{threshold} = \mu + 0.5\sigma
    \end{equation}
    其中 $\mu$ 为信号均值,$\sigma$ 为标准差。
    
    \item \textbf{设置最小峰值间距}:
    \begin{equation}
        d_{min} = \frac{f_s}{f_{max}} = \frac{100}{3} \approx 33 \text{ 样本}
    \end{equation}
    其中 $f_s$ 为采样率,$f_{max}$ 为最大步频。
    
    \item \textbf{计算峰值突出度}:
    \begin{equation}
        \text{prominence} = 0.3\sigma
    \end{equation}
    
    \item \textbf{峰值检测}:使用 \texttt{scipy.signal.find\_peaks} 函数,满足:
    \begin{itemize}
        \item 幅值 $>$ 阈值
        \item 与相邻峰值距离 $\geq d_{min}$
        \item 突出度 $\geq$ prominence
    \end{itemize}
    
    \item \textbf{计数}:峰值数量即为步数
\end{enumerate}

\subsection{数学表达}
对于信号 $x[n]$,若满足以下条件则 $n$ 为峰值点:
\begin{equation}
    x[n] > x[n-1] \land x[n] \geq x[n+1] \land x[n] > \text{threshold}
\end{equation}

%========================================
\section{方法二:过零检测法 (Zero-Crossing Detection)}
%========================================

\subsection{算法原理}
步态信号具有周期性,去除直流分量后,信号会周期性地穿过零点。通过检测从负到正的过零次数来估计步数。

\subsection{算法步骤}

\begin{enumerate}
    \item \textbf{去除直流分量}:
    \begin{equation}
        x'[n] = x[n] - \bar{x}
    \end{equation}
    其中 $\bar{x} = \frac{1}{N}\sum_{i=0}^{N-1}x[i]$
    
    \item \textbf{设置最小过零间隔}:
    \begin{equation}
        T_{min} = 0.3 \times f_s = 30 \text{ 样本}
    \end{equation}
    
    \item \textbf{检测上升沿过零点}:
    \begin{equation}
        \text{crossing}[n] = 
        \begin{cases}
            1 & \text{if } x'[n-1] < 0 \land x'[n] \geq 0 \\
            0 & \text{otherwise}
        \end{cases}
    \end{equation}
    
    \item \textbf{应用时间约束}:相邻过零点间隔需 $\geq T_{min}$
    
    \item \textbf{计数}:有效过零点数量即为步数
\end{enumerate}

\subsection{数学表达}
过零检测器的输出:
\begin{equation}
    ZC = \sum_{n=1}^{N-1} \mathbf{1}[x'[n-1] \cdot x'[n] < 0 \land x'[n] > x'[n-1]]
\end{equation}

%========================================
\section{方法三:自相关函数法 (Autocorrelation)}
%========================================

\subsection{算法原理}
步态信号具有周期性,其自相关函数在周期对应的时延处会出现峰值。通过分析自相关函数估计步态周期,进而计算步数。

\subsection{算法步骤}

\begin{enumerate}
    \item \textbf{去除直流分量}:
    \begin{equation}
        x'[n] = x[n] - \bar{x}
    \end{equation}
    
    \item \textbf{计算自相关函数}:
    \begin{equation}
        R_{xx}[\tau] = \sum_{n=0}^{N-1-\tau} x'[n] \cdot x'[n+\tau]
    \end{equation}
    
    \item \textbf{归一化}:
    \begin{equation}
        R'_{xx}[\tau] = \frac{R_{xx}[\tau]}{R_{xx}[0]}
    \end{equation}
    
    \item \textbf{确定搜索范围}:
    \begin{align}
        \tau_{min} &= \frac{f_s}{f_{max}} = \frac{100}{3.0} \approx 33 \text{ 样本} \\
        \tau_{max} &= \frac{f_s}{f_{min}} = \frac{100}{0.5} = 200 \text{ 样本}
    \end{align}
    
    \item \textbf{找到第一个主峰}:在 $[\tau_{min}, \tau_{max}]$ 范围内找第一个峰值,对应步态周期 $T_s$
    
    \item \textbf{计算步频}:
    \begin{equation}
        f_{step} = \frac{f_s}{T_s}
    \end{equation}
    
    \item \textbf{估计步数}:
    \begin{equation}
        N_{steps} = \lfloor D \times f_{step} \rfloor
    \end{equation}
    其中 $D$ 为信号总时长(秒)
\end{enumerate}

\subsection{数学表达}
自相关函数反映信号与其延迟版本的相似度:
\begin{equation}
    R_{xx}[\tau] = E[x[n] \cdot x[n+\tau]] = \int_{-\infty}^{\infty} x(t)x(t+\tau)dt
\end{equation}

%========================================
\section{方法比较}
%========================================

\subsection{优缺点对比}

\begin{table}[h]
\centering
\caption{三种计步方法优缺点比较}
\begin{tabular}{@{}p{2.5cm}p{5cm}p{5cm}@{}}
\toprule
\textbf{方法} & \textbf{优点} & \textbf{缺点} \\
\midrule
\textbf{峰值检测法} & 
\begin{itemize}[leftmargin=*,nosep]
    \item 直观易理解
    \item 计算效率高
    \item 适合规律行走
    \item 可精确定位每一步
\end{itemize} & 
\begin{itemize}[leftmargin=*,nosep]
    \item 对阈值敏感
    \item 易受噪声干扰
    \item 不规则步态误检多
\end{itemize} \\
\midrule
\textbf{过零检测法} & 
\begin{itemize}[leftmargin=*,nosep]
    \item 不依赖幅值阈值
    \item 对信号幅值变化鲁棒
    \item 实现简单
    \item 计算开销小
\end{itemize} & 
\begin{itemize}[leftmargin=*,nosep]
    \item 对噪声敏感(产生虚假过零)
    \item 需要良好的滤波
    \item 难以处理非周期信号
\end{itemize} \\
\midrule
\textbf{自相关函数法} & 
\begin{itemize}[leftmargin=*,nosep]
    \item 抗噪声能力强
    \item 可估计步频
    \item 适合稳态行走分析
    \item 对局部异常不敏感
\end{itemize} & 
\begin{itemize}[leftmargin=*,nosep]
    \item 计算复杂度较高 $O(N^2)$
    \item 需要较长数据段
    \item 步频变化时精度下降
    \item 无法精确定位每一步
\end{itemize} \\
\bottomrule
\end{tabular}
\end{table}

\subsection{适用场景}

\begin{itemize}
    \item \textbf{峰值检测法}:适合室内规律行走、需要精确定位每步时刻的应用
    \item \textbf{过零检测法}:适合对实时性要求高、计算资源受限的嵌入式设备
    \item \textbf{自相关函数法}:适合离线分析、步态周期估计、稳态长时间行走
\end{itemize}

\subsection{计算复杂度}

\begin{table}[h]
\centering
\caption{计算复杂度比较}
\begin{tabular}{@{}lcc@{}}
\toprule
\textbf{方法} & \textbf{时间复杂度} & \textbf{空间复杂度} \\
\midrule
峰值检测法 & $O(N)$ & $O(1)$ \\
过零检测法 & $O(N)$ & $O(1)$ \\
自相关函数法 & $O(N^2)$ 或 $O(N\log N)$* & $O(N)$ \\
\bottomrule
\end{tabular}
\end{table}
\small{*使用FFT加速}

%========================================
\section{总结}
%========================================

三种方法各有特点:
\begin{enumerate}
    \item \textbf{峰值检测法}:最常用,平衡了准确性和计算效率
    \item \textbf{过零检测法}:最简单,适合资源受限场景
    \item \textbf{自相关函数法}:最鲁棒,适合需要步频分析的场景
\end{enumerate}

实际应用中,可根据硬件资源、实时性要求和精度需求选择合适的方法,或结合多种方法进行融合以提高鲁棒性。

\end{document}
